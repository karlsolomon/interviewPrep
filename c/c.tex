\documentclass{article}

\usepackage[a4paper,margin=1in]{geometry}
\usepackage[T1]{fontenc}
\usepackage{xcolor}
\usepackage{tgcursor}
\usepackage{listings}
\usepackage{hyperref}

\lstdefinestyle{cpp}{
    language=C++,
    basicstyle=\ttfamily\small,
    keywordstyle=\color{blue},
    stringstyle=\color{red},
    commentstyle=\color{gray},
    morecomment=[l][\color{magenta}]{\#},
    numbers=left,
    numberstyle=\tiny\color{gray},
    stepnumber=1,
    numbersep=8pt,
    showstringspaces=false,
    breaklines=true,
    frame=single,
    rulecolor=\color{black},
    backgroundcolor=\color{white},
    tabsize=2,
    captionpos=b
}


\begin{document}
\fontfamily{qcr}\selectfont
\title{Interview Review Chart}
\author{Karl Solomon}
\maketitle
\tableofcontents
\part{C}
\section{Preprocessor}
    \begin{lstlisting}[style=cpp]
# // stringizes the macro parameter
#define stringify(x) #x
#define foo 1
stringify(foo) // --> evaluates to "foo", NOT "1"
\end{lstlisting}

    \begin{lstlisting}[style=cpp]
## // concatenates the macro parameter
#define COMMAND(NAME)  {#NAME, NAME ## _command}
struct command commands[] = {
  COMMAND(quit), // equivalent to {quit_command}
  COMMAND(help), // equivalent to {help_command}
}
\end{lstlisting}

    \begin{itemize}
      \item predefined macros
        \begin{itemize}
          \item \_\_FILE\_\_
          \item \_\_LINE\_\_
          \item \_\_DATE\_\_
          \item \_\_TIME\_\_
          \item \_\_STDC\_VERSION\_\_
          \item \_\_cplusplus
        \end{itemize}
      \item item2
      \item item3
      \item item4
    \end{itemize}

\section{Peripherals}
    \begin{itemize}
      \item \textbf{I2C}\\
        SDA is data, SCL is clock. PURs typically in the 1-4.7k range. Too weak = slow comm and errors. Clocks are usually 100k-1MHz. Addr can be 7 or 10 bit. This is rate-limiter for number of slaves, though line impedance would increase for each slave. Here are some use usage examples:
        \begin{enumerate}
          \item Master sends START and slave Addr
          \item Master sends data to slave
          \item Master terminates with a STOP
        \end{enumerate}
        \begin{enumerate}
          \item Master sends START and slave Addr
          \item Master sends data to slave
          \item Master sends repeatedSTART and either sends more data to slave or receives data from slave.
          \item Master sends STOP
        \end{enumerate}
      \item \textbf{SPI}\\
        Serial Peripheral Interface (SPI) is a synchronous serial communication protocol used for short-distance communication, primarily in embedded systems. It uses a master-slave architecture with a single master and multiple slaves. Communication is full-duplex, and it requires four wires: MOSI, MISO, SCLK, and SS.
      \item \textbf{UART}\\
        Universal Asynchronous Receiver-Transmitter (UART) is a hardware communication protocol that uses asynchronous serial communication with configurable speed. It is commonly used for communication between microcontrollers and peripherals. UART requires only two wires: TX (transmit) and RX (receive).
      \item \textbf{USB}\\
        Universal Serial Bus (USB) is an industry-standard for short-distance digital data communications. It supports plug-and-play installation and hot swapping. USB is used for connecting peripherals such as keyboards, mice, printers, and external storage devices to computers.
      \item \textbf{HDMI}\\
        High-Definition Multimedia Interface (HDMI) is a proprietary audio/video interface for transmitting uncompressed video data and compressed or uncompressed digital audio data from an HDMI-compliant source device to a compatible display device. It is commonly used for connecting devices like TVs, monitors, and projectors.
    \end{itemize}
\end{document}
\enddocument
